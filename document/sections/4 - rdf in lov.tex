\section{RDF in LOV}

\subsection{Alcune Knowledge Base importanti}
\subsubsection{Wikidata}
In Wikidata ogni \textbf{elemento} è contraddistinto dal proprio codice univoco, gli elementi rappresentano i vari elementi del sapere umano. Ogni elemento ha una pagina nel namespace con prefisso Q seguito da un numero progressivo. Ogni elemento a sua volta contiene una serie di \textbf{proprietà} caratteristiche dell'elemento, come la capitale o la bandiera per uno Stato. Anche le proprietà sono identificate come gli elementi.

\subsubsection{DBpedia}
Ogni elemento di Wikipedia ha l'analogo riferimento su Wikipedia.

\subsubsection{Wikidata e DBpedia}
Wikidata e DBpedia hanno alcune caratteristiche simili:
\begin{itemize}
	\item Entrambe pubblicano dati in formato RDF;
	\item Entrambe offrono risorse attraverso URI standard e definiscono ontologie;
	\item Entrambe sono linkate ad altri dataset.
\end{itemize}
Ma anche alcune importanti differenze:
\begin{itemize}
	\item la conoscenza di DBpedia è derivata, quella di Wikidata è stata creata in modo collettivo (crowdsourcing);
	\item DBpedia estrae dati da wikipedia, Wikidata fornisce i data a Wikipedia;
	\item DBpedia aggiunge una struttura ai dati di Wikipedia, Wikidata è già nativamente strutturata;
	\item DBpedia è vecchia, Wikidata è recente.
\end{itemize}

\subsubsection{GDELT project}
Un ricchissimo database globale che monitora le trasmissioni televisive, la stampa etc. giornalmente, in un numero esteso di lingue. Può essere utilizzato per esempio per stilare dati riguardanti determinati paesi (situazioni di guerra, azioni politiche, crimini etc.). Ha un knowledge graph di riferimento. Il progetto dispone di diversi servizi.

\subsubsection{vCard}
Un'ontologia che trae spunto da due che già esistono (FOAF e una focalizzata sulle organizzazioni (Organisation Ontology)).

\subsubsection{FOAF}
Uno dei primi vocabolari utilizzati per esprimere il concetto di amicizie e legami tra persone.

\subsubsection{AIISO}
Academic Institution Internal Structure Ontology dispone di classi e proprietà per descrivere la struttura organizzativa di istituzioni accademiche.

\subsubsection{Time ontology}
OWL-Time viene utilizzata per descrivere concetti temporali.

\subsubsection{DOAC e DOAP}
DOAC (Description Of a Career) e DOAP (Description Of A Project) sono due estensioni di FOAF per descrivere rispettivamente CV e progetti software.

\subsubsection{Creative Commons}
Utilizzata per stabilire i diritti di autore di una risorsa.

\subsubsection{GoodRelations}
Ha senso in contesto di eCommerce. Un vocabolario per la descrizione di risorse impiegate nei comuni diti di commercio elettronico.
