\section{Introduzione}
\subsection{Knowledge Graph}
Si tratta di un una struttura dati (grafo) utilizzato per rappresentare informazioni. Ogni nodo rappresenta una \textbf{risorsa} o un \textbf{valore}, mentre gli archi rappresentano le \textbf{relazioni}.\newline
Una tipla nodo-arco-nodo rappresenta un \textbf{Fatto}. Un Knowledge Graph quindi rappresenta la conoscenza coe un insieme di fatti connessi. Una struttura di questo tipo è necessaria per mantenere le informazioni in modo che possano essere salvati sia i concetti comuni che quelli distinti.
\begin{info}[Esempio]
	Plutone il 24 agosto 2006 è diventato un pianeta nano, ma i concetti variano anche in funzione dei campi di interessa. Ad un astronomo interessano certe informazioni, ad un astrologo altre. Come possiamo mantenere tutte le informazioni?
	\begin{itemize}
		\item Mantenere un unico "governatore unico" del concetto di Plutone che esprime qual è la visione "preferita" di quell'ambito. In questo caso vincerebbe una delle due strade tra astronomia e astrologia. Questo equivale ad un amministratore globale del dato, e questo è impraticabile.
		\item Permettere a ciascuno di poter esprimere la propria opinione, rimuovendo controllori e amministratori del DB lasciando libertà totale nell'esprimere qualsiasi cosa. Il problema in questo caso è che il consumatore deve trovare ogni possibile punto di vista del concetto che vuole esprimere.
	\end{itemize}
	Utilizzando il grafo di conoscenza manteniamo la libertà, ma utilizziamo un modello che può aiutarci a mantenere tutte le informazioni che vogliamo.
\end{info}
Quello verso cui ci muoviamo è un grafico simile a questo:
\centeredImage{document/img/graphmonna.PNG}{Esempio di Knowledge Graph desiderato}{0.6}
(In questo esempio la dimensione dei nodi non ha alcun significato).\newline
Per arrivare a questa modellazione dobbiamo usare concetti di programmazione orientata agli oggetti, ad esempio le gerarchie.\newline
Per esempio, a casa Unibo, un professore può essere di tantissimi tipi: ricercatore, associato, associato confermato etc...\newline
La parte fondamentale è mettersi d'accordo sulla definizione dei singoli concetti, di solito esistono organizzazioni apposite per le definizioni ad altri livelli (vedi esempio dei pianeti).\newline

\begin{itemize}
	\item \textbf{Tecnicità}:
	\begin{itemize}
		\item esistono formati standard per salvare i dati e usarli in più grafi (compresi i tipi di dati come stringhe, numeri etc.);
		\item esistono linguaggi e database appositi per la gestione del dato;
		\item esistono API per la maggior parte dei linguaggi.
	\end{itemize}
	\item \textbf{Semantica}:
	\begin{itemize}
		\item Esistono vocabolari per identificare univocamente risorse e le loro relazioni, grazie a repositories immense;
		\item Possibile utilizzarli per applicazioni di Machine Learning;
	\end{itemize}
	\item \textbf{Formalità}:
	\begin{itemize}
		\item I linguaggi seguono regole ben definite ed esistono motori di inferenza, o sistemi esperti (es: il sistema che giocando a scacchi ha battuto il campione mondiale);
	\end{itemize}
\end{itemize}
\newpage
